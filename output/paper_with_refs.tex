
\title{Personai: Chain of Thought for Mitigating Personas in Large Language Models}

\begin{abstract}
Large Language Models (LLMs) trained on potentially contaminated data may exhibit harmful behaviors at inference time due to the activation of personas. Existing approaches, such as data filtering at the node level, have limitations in terms of accuracy and scalability. We propose \texttt{\textsc{Personai}}, a novel approach for detecting persona influences in a way that can also mitigate them. In response to the challenge of detecting harmful personas, we suggest a new paradigm of persona detection at paths (rather than data nodes), motivated by two key findings: (1) LLMs exhibit a sensitivity to persona-activating prompts (PAEs) and (2) there is a phenomenon called the \textit{enchainment effect} of sycophancy. Leveraging these, \texttt{\textsc{Personai}} implements a progressive PAE search and a retrieval method that can estimate the degree of sycophancy towards each prompt. Evaluation shows that \texttt{\textsc{Personai}} significantly outperforms traditional approaches in both scalability and accuracy, with compatibility verified across various LLMs and datasets. 
\end{abstract}

\section{Introduction}



\label{sec:Introduction}

Large Language Models (LLMs) have shown some common behaviors that may cause various social impacts, such as sycophancy toward potentially harmful content in the training data~\cite{malmqvist2024}, which can result in LLMs discriminating against minorities~\cite{buyl2024} and can be activated~\cite{pellert2024}. Therefore, the impact of data Filtering~\cite{bai2024} and trustworthy enhancement~\cite{chen2025} in eliminating such harmful personas in training data has been intensively investigated. However, although the harmful personas may be suppressed during training, they still remain in the model due to limited data filtering capabilities and can thus be adversarially reactivated~\cite{jiang2024}.

Recently, the impact of personas in LLMs, in both negative~\cite{yu2024} and positive~\cite{ye2024} ways, has garnered attention. However, the impact of persona influences in LLMs---especially, the potential for adversaries to exploit them to elicit unintended responses from LLMs~\cite{xu2025}---is still under research. In this work, we introduce a novel approach for detecting and mitigating persona influences in a way that can also reveal potential safety risks in LLMs.

Sycophancy refers to the tendency of LLMs to align with users due to the contamination of training data~\cite{malmqvist2024}. A simple baseline to mitigate sycophancy is based on the observation that LLMs may generate different responses to prompts beginning with the PAE ``\texttt{As an}, e.g., "As an AI."~\cite{yu2024}. Built upon this, we design a data filtering algorithm that re-ranks data samples in the training dataset based on the differences in outputs between the prompts "As an AI" and "As a human," which significantly improves the effectiveness of data filtering. However, the scalability of such algorithms is constrained due to the time required for running the inference of two prompts for each data sample. 

More recently, the approach of using synthetic prompts to identify harmful data has also been demonstrated by~\citet{buyl2024}, who reveal the potential of sycophantic behaviors in LLMs. However, the detection of harmful data is still challenging. Our work reveals two key findings to facilitate persona detection with respect to sycophancy: (1) The probability of a sample containing harmful content increases as the divergence of two responses from different PAEs (e.g., "As a human" vs. "As an AI") increases and (2) there is a phenomenon known as the \textit{enchainment effect of sycophancy}: due to the contamination of data, LLMs demonstrate more and deeper sycophantic behaviors after being, essentially, taught to sycophantic by other sycophantic behaviors. More specifically, we find that the more a model demonstrates sycophancy, the more it can be guided to learn other sycophancy behaviors, resulting in a snowball effect. More importantly, this effect can be exploited to retrieve harmful data in the dataset in a progressive manner, which can be used to mitigate sycophancy at a much higher efficiency.

Leveraging this, we propose a novel approach for persona detection named \texttt{\textsc{Personai}}. Specifically, we devise a progressive PAE prompt generation algorithm that can iteratively expand a pool of PAEs to significantly enhance data filtering efficiency. More importantly, we formulate persona detection as a prompt retrieval task, which can mitigate harmful personas without the need for data filtering. Moreover, the persona detection problem can be formulated as a prompt retrieval problem, which can be solved using existing training-free prompt retrieval approaches. \texttt{\textsc{Personai}} can thus be highly scalable. Our empirical analysis demonstrates the superior performance of \texttt{\textsc{Personai}}, outperforming baseline methods in both effectiveness and scalability.

In general, the contributions of this work are three-fold. (1) We reveal two key findings that can facilitate persona detection with respect to sycophancy. (2) We propose a novel approach \texttt{\textsc{Personai}}, a framework that can detect and mitigate personas in a manner that can significantly improve both scalability and accuracy. (3) The results show that \texttt{\textsc{Personai}} consistently outperforms baseline methods in terms of both scalability and accuracy, with performance verified across different LLMs and datasets.  



\section{Preliminaries and Related Works}

\label{sec:preliminaries}
\textbf{Prompt Retrieval.}
Prompt retrieval approaches have been explored in various fields, including NLP~\cite{gao2023}, with one representative method being the CoReevo~\cite{gao2023} approach. CoReevo implements a genetic search for a prompt that can make a pretrained LLMs to output a target response. In the mixture of prompt retrieval and fine-tuning, in-context learning, and prompt engineering, a simple and intuitive method is to use manually designed PAEs~\cite{yu2024} to detect harmful data behaviors~\cite{buyl2024}. These methods can also be used for persona detection, but they are still insufficient for handling large-scale datasets due to the low scalability of the genetic search. In comparison, we propose a progressive PAE generation algorithm that can efficiently expand the pool of PAEs, which can be used to improve the scalability of data filtering with PAEs.

\textbf{Persona Detection.}
Due to the influence of existing data nodes with malicious content, LLMs may exhibit unintended behaviors, such as hallucinations, biases, and limited reasoning capabilities, which have been extensively investigated in the literature~\cite{he2023, hu2024, tsai2024, wang2025, xhonneux2024, hu2025}. However, most works focus on detecting and eliminating such harmful data during the training stage. For example, data filtering at the level of data nodes has been an effective approach that is widely adopted for trustworthy AI. This is also the technique used in the only relevant work that detects sycophancy using data nodes. More specifically, it is based on the observation that sycophantic samples tend to have similar responses to prompts beginning with "As a [prompt]," such as human, AI, and customer. Using this observation, a simple baseline method can be designed as follows. Suppose the score function $S(\cdot)$ for a given model $\mathcal{M}$ is defined as $S(p, x, y) = 1.0$ if the response to the prompt $p$ for a sample $x$ outputs $y$, otherwise $S(\cdot) = 0.0$. Then, for each $x$ in the training dataset $\mathcal{D}$, we can calculate
\begin{equation}
\label{equ:baseline}
    \mathbb{E}\left[S("As a [P]," x, y)\right],
\end{equation}
where $P$ is sampled from a set of categories in a sample distribution $Pr(P)$, e.g., $\mathcal{P}=\{$AI, human, customer, etc.$\}$. The worst response rate of sample $x$ is then calculated by taking a weighted average over all $p\in \mathcal{P}$. The dataset can be reordered based on the worst response rate. Although this can effectively detect sycophantic samples, it requires the inference of two prompts for each sample, which may be time-consuming for large-scale datasets. In comparison, \texttt{\textsc{Personai}} formulates the persona detection problem as a prompt retrieval problem, which can handle large datasets in a much more efficient manner.

\textbf{Toxicity Detection.}
While there have been various techniques designed to detect toxic data, such as those by~\citet{bai2024, zhang2024}, they are still irrelevant to data samples without direct references to toxic content. This makes them less effective for detecting data that leads to persona influences. Moreover, most toxicity detection methods are trained to identify specific types of data as toxic. They thus require a significant amount of human involvement in defining the criteria for filtering and in annotating data, which is infeasible for datasets at the internet scale. In comparison, \texttt{\textsc{Personai} is data-free, requiring no human definition for personas, data annotation, or type-specific training.



\section{Method}

\label{sec:Method}
\subsection{Two Key Findings}
\label{subsec:finding}

The persona that a model learns from the training dataset may influence the model's outputs in various and complex ways. In this work, we focus on sycophancy, by which we can design a simple baseline method to detect sycophantic samples, as described in Section~\ref{sec:preliminaries}. Due to the data contamination, a model may be activated to exhibit persona influences that it would not had the data been cleaned in the first place. The model may generate harmful content when prompted with a PAE that it is not familiar with, as such a persona is likely to be contained in the training data. For example, the persona of "As an AI" may have been learned from the training data, resulting in the model generating harmful content when prompted with "As an Assistant". In response to this, we design a method to detect sycophancy-related data when the model encounters PAEs that are not familiar with. We empirically find that, for a data sample $x$, its tendency to generate harmful content is proportional to the sample divergence when responding to familiar PAEs, e.g., "As an AI" and "As a Human". 

More specifically, we empirically find two key factors for detecting samples with harmful personas. (1) \textbf{Sensitivity to PAEs.} As shown in Figure~\ref{fig:find1}, there is a correlation between the tendency to generate harmful content (i.e., sycophancy) and the divergence of the response when prompted with a familiar PAE. We define the degree of sensitivity to PAEs (denoted as \textit{sensitivity}) as 
\begin{equation}
    s = \mathbb{E}\left[D(\mathcal{M}(p_1, x, y_1), \mathcal{M}(p_2, x, y_2))\right]_{p_1, p_2 \sim \mathcal{P}_s},
    \label{equ:sensitivity}
\end{equation}
where $\mathcal{M}(\cdot)$ denotes the response of the LLMs with one prompt and one sample paired with the desired response, $D(\cdot)$ is a divergence function between two responses, and $\mathcal{P}_s$ is a set of familiar PAEs. To be specific, the divergence $D(\cdot)$ can be (1) equipped with the cosine similarity function, i.e., $D(t_1, t_2)=1-CS(t_1, t_2)$; (2) $D(\cdot)$ can also be the Divergence calculator with respect to a reference sample, i.e., $D(t_1, t_2) = \max(t_1,t_2) - \min(t_1,t_2)$, in which all tokens that exist in only one of the two responses are considered as $1.0$. Furthermore, the sensitivity score for each $x$ can be calculated by taking a weighted average over all $(y_1, y_2)$ in the sample pairs $\mathcal{Y}_s$, defined over $\mathcal{P}_s$ PAEs. 

In addition, we also find that (2) \textbf{Sycophancy Enchainment Effect} as follows. The model may learn persona influences in an enchained manner. More specifically, we observe that a model trained on sycophantic samples in the training dataset may exhibit a greater tendency of sycophancy after being taught other sycophantic samples. For a data sample $x$, its degree of sycophancy can be calculated by taking a \textit{weighted} average over $(y_1, y_2)$ in the sample pairs $\mathcal{Y}_g$, i.e., 
\begin{equation}
    g = \mathbb{E}_ {(y_1, y_2) \sim \mathcal{Y}_g}\left[D(\mathcal{M}(p_1, \mathcal{M}(p_2, x, y_2), y_1)\right]_{p_1, p_2 \sim \mathcal{P}_g},
\end{equation}
where $g$ represents the degree of sycophancy, $p_1 \in \mathcal{P}_g$ is a sycophantic prompt, and $p_2 \in \mathcal{P}_g$ is a guide prompt. The demonstration of sycophancy is exhibited when encountering $p_1$, while the guide prompt $p_2$ is used to teach the persona that can be activated by $p_1$. More specifically, t This is shown by the blue arrow $\sim \mathcal{P}_g$. More details of the enchainment effect are demonstrated in Figure~\ref{fig:find2}.

\begin{figure}[!t]
\centering
\includegraphics[width=0.33\textwidth]{figs/find1.png}\!\!\!\!\!\
\includegraphics[width=0.33\textwidth]{figs/find2.png}\!\!
\includegraphics[width=0.33\textwidth]{figs/find3.png}
\caption{Three figures are created using \texttt{Llama2-7b}. (Left) Tendency of sycophancy versus various familiar PAEs. (Middle) Tendency of sycophancy versus various guide PAEs. (Right) Tendency of sycophancy versus guide and familiar PAEs. Familiar PAEs include different types of sycophantic prompts. Guide PAEs are selected based on the highest score in the calculation. Syekphancy is evaluated under different familiar PAEs.}
\label{fig:find1}
\label{fig:find2}
\label{fig:find3}
\end{figure}

The tendency of sycophancy shown in Figure~\ref{fig:find1}, i.e., the divergence between samples $x$ increases as the sample tendency to sycophancy increases. This shows that the sensitivity to unfamiliar PAEs can be used to detect samples with harmful personas. In addition, the left and middle figures in Figure~\ref{fig:find1} demonstrate that the tendency of sycophancy increases as we use PAEs that cause higher degrees of sycophancy. More importantly, in the right figure, we observe that sycophancy increases significantly after encountering guide samples, shown by the blue bar. This phenomenon demonstrates the effect of enchainment, i.e., the more the model demonstrates sycophancy, the more it can be guided to learn other sycophancy behaviors. However, due to the limited number of guide samples that a model can encounter, the effect may saturate when there are too many guide samples.

\subsection{\texttt{Personai}}

\textbf{Overview}. \texttt{\textsc{Personai}} leverages the above two findings to detect samples with persona influences. It can also mitigate such persona influences without data filtering. More specifically, \texttt{\textsc{Personai}} adopts a prompt retrieval approach to identify sycophantic samples. In general, using the aforementioned two findings, we can design a simple baseline method to detect sycophantic samples, which is presented in Algorithm~\ref{alg:algorithm1}. The algorithm is data filtering based on the divergence when encountering both familiar PAEs and guide PAEs. However, the algorithm is not scalable as it requires the inference of multiple PAEs for each sample. In response to this, we use a progressive PAE generation algorithm that can iteratively expand the pool of PAEs. This algorithm exploits the enchainment effect of sycophancy and can efficiently expand the pool of guide PAEs to accelerate the retrieval. We then adopt a prompt retrieval method to detect samples with persona influences, which does not require multiple inferences for each sample, thus is scalable for large datasets. 
\begin{algorithm}[!t]
\setstretch{1.25}
\SetNoFillInSpace
\SetAlgoVspace{0.1 cm}
\SetNColorOnly
  \LinesNumbered
  \caption{ A simple baseline method for sycophancy detection.}
  \label{alg:algorithm1}
  \KwIn{Data set $\mathcal{D}=\{x\}$, PAE sets $\mathcal{P}_s$, $\mathcal{P}_g$}
    \SetCommentSty{myComment} 
    \Comment*[r]{Familiar PAEs}
     \ForEach{$x \in \mathcal{D}$}{   
        $s[ x] = 0$, temp$\leftarrow \mathcal{M}(p_1, x, y_1)$
        
        \ForEach{$p_2 \in \mathcal{P}_s$}{
            temp$\leftarrow\mathcal{M}(p_2, x, y_2)$
            
            $s[x] += D(\temp, temp)$
        }
    }
    \Comment*[r]{Guide PAEs}
    \ForEach{$x \in \mathcal{D}$}{
        temp$\leftarrow\mathcal{M}(p_1, x, y_1)$
        
        \ForEach{$p_2 \in \mathcal{P}_g$}{
            temp$\leftarrow\mathcal{M}(p_2, \temp, y_2)$
            
            $s[x] += D(\temp, temp)$
        }
    }
\KwRet{$\mathcal{D}=\{s[x]\}$}
\end{algorithm}

\textbf{Progressive PAE Generation.} The pool of PAEs can be expanded by the enchainment effect of sycophancy. In general, given a seed PAE for sycophancy generation, each new PAE is generated based on the current pool of PAEs $\mathcal{P}$, a guide sample set $\mathcal{D}_g$, a generation step $\alpha$, and a top-$k$ selection, denoted as $Init(\cdot,\cdot,\cdot)$, which can be presented in Algorithm~\ref{alg:algorithm2}. More specifically, to generate a new PAE, we sample a guide prompt from the current pool of PAEs, as well as a guide sample from the guide sample set. Guiding with the sampled PAE and thus activating the sampled persona in the guide sample can thus result in a new PAE. Graciously, as we desire to expand the pool with diverse PAEs, we then generate new PAEs with different guide samples, each persona activated by a different PAE. Guidedly, guide samples will generate PAEs that are diverse in distribution. Therefore, we can continuously grow the pool of PAEs in each step. In addition, the new PAEs will be distinguished from guide PAEs by the guide sample annotation, as detailed in Algorithm~\ref{alg:algorithm2}.

In practical use, the guide sample set can be annotated using either human effort or existing PAEs. For example, we can leverage existing PAEs to generate new PAEs. In this way, the method requires no human effort, which is highly scalable. In addition, the method is also not restricted to any specific guide sample annotation. For example, we may use existing enchainment data in the training data, e.g., customer vs. AI, to annotate new guide PAEs. In this way, we can thus customize the guide sample set to improve the diversity of the expanded PAEs. Moreover, when the guide samples are annotated, the method no longer requires a pretrained LLMs that contains harmful behaviors. More specifically, the only model required is only used for guide sample annotation, not for the generation of PAEs. This also improves the performance of the expansion of PAEs, as the calculation of PAEs may contain errors. 

\begin{algorithm}[!t]
\setstretch{1.20}
\SetNoFillInSpace
\SetAlgoVspace{-0.1 cm}
\SetNColorOnly
  \LinesNumbered
  \caption{Progressive PAE Generation.}
  \label{alg:algorithm2}
  \KwIn{Guide sample set $\mathcal{D}_g$, PAE pool $\mathcal{P}$, expansion step $\alpha$}
    \Comment{Initialize guide prompt $\mathcal{P}_g \leftarrow \{p_g^*\}$}
    
    \Ind{while}{$\alpha>0$}{
        
        \For{each guide prompt $p_g^* \in \mathcal{P}_g$}{
            Sample a guide sample $x_g^* \sim \mathcal{D}_g$
            
            \Ind{if}{ $x_g^*$ corresponds to $p_g^*$}{
                Sample $\alpha$ data samples $\hat{x}_0,..,\hat{x}_{\alpha}$ from $\mathcal{D_g}$
                
                \ForEach{$\hat{x}$}{
                    Generate a new PAE $\hat{p}$, i.e., $p_g^* \oplus \hat{x}$ $\leftarrow \{\mathcal{M}(p_g^*, \hat{x}, \hat{x})\}$
                
                }
                
                Take a step random sampling over $\alpha$ new PAEs $\hat{p}_1, ..., \hat{p}_{\alpha}$
                
                Add $\{\hat{p}_1, ..., \hat{p}_{\alpha}\}$ to $\mathcal{P}_g$
            }
        }
    }
\KwRet{$\mathcal{P}$}
\end{algorithm}

\textbf{Prompt Retrieval for Persona Detection.} Using the expanded PAEs, we can design a simple baseline method for sycophancy detection, which is presented in Algorithm~\ref{alg:algorithm3}. In general, sycophancy can be detected based on the divergence when encountering unfamiliar PAEs. Using the sensitivity score presented in Equation~\ref{equ:sensitivity}, for each sample $x \in \mathcal{D}$, we calculate the pair-wise response differences when encountering each PAE in the pool. The differences can be used to calculate the score $s[x]$ for each $x$. In practical use, additional guide prompts can be explored to improve the degree of sycophancy revealed by each PAE. More specifically, considering that the PAE pool $\mathcal{P}$ may be unevenly distributed, guide prompts can serve as bridges to fill the gaps between different types of PAEs. Therefore, for each PAE $p \in \mathcal{P}$, we can use the corresponding guide sample to generate a new PAE, which can be added to $\mathcal{P}$.

\begin{algorithm}[!t]
\setstretch{1.20}
\SetNoFillInSpace
\SetAlgoVspace{-0.1 cm}
\caption{Persona Detection: A prompt retrieval approach.}
\label{alg:algorithm3}
\SetNoFillComment
\SetNColorOnly
\LARGE
\linesnumbered
\SetInd{1}{here}
\SetInd{2}{there}%% after first line
\SetInd{3}{overhere} %% at outmost level
\SetInd{4}{innerhere}
\KwIn{Dataset $\mathcal{D}$, Pool of PAEs $\mathcal{P}$}
    \Comment{Initialization}
    
    \ForEach{$x \in \mathcal{D}$}{
        
        $s[x] = 0$
        
    }        
    
    \ForEach{$p \in \mathcal{P}$}{
        
        $temp \leftarrow \mathcal{M}(p, x, y)$
        
        \ForEach{$p_g \in \mathcal{P}_{g}[p]$}{
            
            $temp_g \leftarrow \mathcal{M}(p_g, x, y)$
            
            $s[x] += D(\temp, temp_g)$
        }
    }

    \KwRet{$\mathcal{D}=\{s[x]\}$}
\end{algorithm}

\begin{figure}[!t]
\centering
\includegraphics[width=0.33\textwidth]{figs/rebuttal/bai_7b.png}\!\!\!\!\!\
\includegraphics[width=0.33\textwidth]{figs/rebuttal/retr_7b.png}\!\!\!\!\!\
\includegraphics[width=0.33\textwidth]{figs/rebuttal/top10_7b.png}
\caption{Left: The sycophancy score of \citet{bai2024}. Middle: The sycophancy score of PAE retrieval. Right: The sycophancy score of top-$10$ PAE retrieval.}
\label{fig:rebuttal-pae}
\end{figure}

\begin{table*}[t]
\centering
\caption{Sycophancy Detection Evaluation. ``Train'' denotes the original dataset, while ``Test'' denotes the dataset evaluated using data filtering. On the Train dataset, we report the F1 score and accuracy evaluated at different thresholds. In Test dataset, we report the highest F1 score and accuracy. For both scenarios, we report the aggregate values calculated across all guide prompts. For Personai, we also report the accuracy of PAE generation for guide expansion. F1/Bedrock results are reported by multiplying the output log probability, following the setting of~\citet{bai2024}.}
\label{tab:main}
\resizebox{\textwidth}{!}{
\begin{tabular}{l|l|ccccc|ccccc|ccccc}
\Xhline{1.0pt}
   &\multirow{2}{*}{\textbf{Guide}} & \multicolumn{5}{c|}{\textbf{Train}} & \multicolumn{5}{c|}{\textbf{Test}} & \multicolumn{5}{c}{\textbf{Top-$100$}} \\
   \cline{3-16}
   & & F1 & Accuracy @$0.1$ & Accuracy @$0.3$ & Accuracy@$0.5$ & Avg. & F1 & Accuracy @$0.1$ & Accuracy @$0.3$ & Accuracy @$0.5$ & Avg. & F1 & Accuracy @$0.1$ & Accuracy @$0.3$ & Accuracy @$0.5$ & Avg. \\ \Xhline{1.0pt}  
\multicolumn{2}{l|}{\texttt{Llama2-7B}} 
& \texttt{0.205} & \texttt{0.089} & \texttt{0.096} & \texttt{0.098} & \texttt{0.102} & \texttt{0.205} & \texttt{0.089} & \texttt{0.097} & \texttt{0.098} & \texttt{0.102} & \texttt{0.247} & \texttt{0.143} & \texttt{0.144} & \texttt{0.146} & \texttt{0.153} \\
\multicolumn{2}{l|}{\texttt{Llama2-13B}} & \texttt{0.208} & \texttt{0.093} & \texttt{0.104} & \texttt{0.109} & \texttt{0.107} & \texttt{0.208} & \texttt{0.093} & \texttt{0.105} & \texttt{0.111} &  \texttt{0.107} & \texttt{0.293} & \texttt{0.183} & \texttt{0.188} & \texttt{0.194} & \texttt{0.203}\\
\multicolumn{2}{l|}{\texttt{Llama3-8B}}  & \texttt{0.26} & \texttt{0.145} & \texttt{0.166} & \texttt{0.179} & \texttt{0.173} & \texttt{0.265} & \texttt{0.148} & \texttt{0.173} & \texttt{0.182} & \texttt{0.173} & \texttt{0.285} & \texttt{0.187} & \texttt{0.191} & \texttt{0.195} & \texttt{0.213} \\
\multicolumn{2}{l|}{\texttt{Llama3-70B}}  & \texttt{0.268} & \texttt{0.162} & \texttt{0.162} & \texttt{0.157} & \texttt{0.162} & \texttt{0.268} & \texttt{0.162} & \texttt{0.162} & \texttt{0.157} & \texttt{0.162} & \texttt{0.268} & \texttt{0.162} & \texttt{0.162} & \texttt{0.157} & \texttt{0.162} \\
\multicolumn{2}{l|}{\texttt{Llama3-80B}}  &  \texttt{0.338} & \texttt{0.238} & \texttt{0.214} & \texttt{0.225} & \texttt{0.231} & \texttt{0.349} & \texttt{0.258} & \texttt{0.226} & \texttt{0.231} & \texttt{0.238} & \texttt{0.343} & \texttt{0.248} & \texttt{0.234} & \texttt{0.238} & \texttt{0.254} \\
\multicolumn{2}{l|}{\texttt{Llama3-70B-chat}}  &  \texttt{0.385} & \texttt{0.308} & \texttt{0.313} & \texttt{0.318} & \texttt{0.322} & \texttt{0.391} & \texttt{0.324} & \texttt{0.335} & \texttt{0.339} & \texttt{0.334}& \texttt{0.403} & \texttt{0.344} & \texttt{0.339} & \texttt{0.344} & \texttt{0.355} \\
\multicolumn{2}{l|}{\texttt{Llama3-80B-instruct}} &  \texttt{0.322} & \texttt{0.258} & \texttt{0.222} & \texttt{0.238} & \texttt{0.247} & \texttt{0.338} & \texttt{0.291} & \texttt{0.246} & \texttt{0.252} & \texttt{0.260} & \texttt{0.317} & \texttt{0.245} & \texttt{0.221} & \texttt{0.234} & \texttt{0.255} \\ \hline \hline
\multirow{2}{*}{\texttt{Mistral-7B}} & \texttt{Caula} & \texttt{0.082} & \texttt{0.048} & \texttt{0.050} & \texttt{0.049} & \texttt{0.058} & \texttt{0.088} & \texttt{0.053} & \texttt{0.055} & \texttt{0.058} & \texttt{0.061} & \texttt{0.085} & \texttt{0.052} & \texttt{0.053} & \texttt{0.055} & \texttt{0.059} \\
& \texttt{Caula2} & \texttt{0.091} & \texttt{0.064} & \texttt{0.059} & \texttt{0.060} & \texttt{0.065} & \texttt{0.084} & \texttt{0.064} & \texttt{0.061} & \texttt{0.065} & \texttt{0.065} & \texttt{0.084} & \texttt{0.064} & \texttt{0.061} & \texttt{0.059} & \texttt{0.065} \\
\hline\hline
\end{tabular}
}
\end{table*}




\section{Experiments}


\subsection{Experimental Setup}
\textbf{Datasets.}
We evaluate our method in both sycophantic and general harmful datasets. Sytcophancy is shown by harmful data, thus we use the two datasets to validate the effectiveness of \texttt{Personai}. The sycophantic dataset is the \texttt{RedQueen} dataset~\cite{jiang2024}, and the general harmful dataset is the \texttt{MMLL}~\cite{gao2023}. More specifically, the \texttt{MMLL} dataset contains data points that cause general harmful behaviors in LLMs, such as discrimination, bias, and bias against minorities. Therefore, although the dataset may not contain pairs of samples that can activate harmful personas in LLMs, we can still use it to test whether \texttt{Personai} works for general harmful data. For more details of these datasets, see in Appendices~\ref{appdx:datasets} and ~\ref{app:dataset}.

\textbf{Baselines.} Data filtering at the level of data nodes has been an effective approach, which we use as a baseline method for sycophancy detection. More specifically, this method is equivalent to Algorithm~\ref{alg:algorithm1}, which can serve as a simple baseline to compare with \texttt{Personai}, with the only difference of not using guide samples. In addition, another relevant method is toxicity detection. We choose \citet{bai2024} as the representative method. This method is different from ours in two aspects. First, it can only target specific harmful content, e.g., ``\texttt{harmful\_prompt\_5}'' and ``\texttt{harmful\_template\_0}'', which are not effective for detecting harmful content unseen in the training data. In comparison, \texttt{Personai} does not require human effort in defining types of harmful data. In addition, \texttt{Personai} uses prompts to detect harmful personas in LLMs, which differs from toxicity detection. Since sycophancy is only detected when a persona is activated, \texttt{Personai} can detect more harmful data, as demonstrated by our performance comparison.

\textbf{Metrics.} We employ precision, recall, and accuracy as primary metrics. To be specific, each data sample is assigned a score for sycophancy. Subsequently, we select various thresholds to classify samples that exceed the threshold as positive, serving as different discriminators. The higher the score, the more likely the sample is considered as containing harmful personas. More specifically, precision, recall, and accuracy correspond to the following three metrics
    \begin{equation}
        Precision = \frac{\lvert\{x|s(x)\geq \delta, y(x)=1\}\rvert}{\lvert\{x|s(x)\geq \delta\}\rvert}, \frac{\lvert\{x|s(x)\geq \delta, y(x)=1\}\rvert}{\lvert\{x\}\rvert} , \frac{\lvert\{x|s(x)\geq \delta\}\rvert}{\lvert\{x\}\rvert},
    \end{equation}
where $\lvert\cdot\rvert$ represents the number of elements contained and $y(x)$ is the ground truth label indicating whether a sample contains harmful personas. More details of metrics can be found in the Appendices~\ref{app:metric}. 

\textbf{Models.}
We demonstrate that \texttt{Personai} is compatible with different LLMs on sycophancy datasets. More specifically, we conduct model-specific experiments on \texttt{Llama2-7B}, \texttt{Llama2-13B}~\cite{llama2}, \texttt{Llama3-8B}, \texttt{Llama3-70B}, \texttt{Llama3-70B-chat}, \texttt{Llama3-80B}, and \texttt{Llama3-80B-instruct}~\cite{llama3}.  As the models vary in size, we show their results for sycophancy detection on \texttt{RedQueen}. In addition, on general harmful datasets, we also evaluate using models from other LLM families, e.g., \texttt{Mistral-7B}. In addition, traditional methods, i.e., \citet{bai2024} and toxicity detection, are demonstrated in two open-weight models \texttt{Cauva1} and \texttt{Cauva2}, trained using the \texttt{LLaMA-2-7B} and the \texttt{Mistral-7B} models. To be specific, the two models are used for data annotation, which is used by these baselines.

\textbf{Guide Samples.}
The guide sample set contains sample pairs that can activate harmful personas in LLMs, with each pair consisting of a guide sample and a target sample. For the guide sample pairs, we use a balanced sampling strategy over the sample pairs in the training set, with an annotation size of $100$ in practical use. More specifically, the guide samples are sampled from the first $5\%$ of the training dataset. For more details of the guide samples used for each setting, see in the Appendices~\ref{app:guide}.

\textbf{Sequences of Guide PAE Generation Steps $\alpha$.} We set $\alpha=20$ for all baselines to ensure a fair comparison. We also test other guide PAE generation steps in the ablation study.

\subsection{Main Results}
\textbf{\texttt{Llama2-7B} and \texttt{Llama2-13B}.} The results on \texttt{RedQueen} of sycophancy detection for \texttt{RedQueen} are shown in Table~\ref{tab:main}. With \texttt{Llama2-7B} as the test model, we achieve average Recall, Accuracy, and Precision of $0.205, 0.089, 0.096, 0.098$, and $0.102$ respectively, significantly outperforming the simple baseline of data filtering at the level of data nodes, with average values of $0.102, 0.045, 0.047, 0.048$, and $0.059$ respectively. In addition, when compared with toxicity detection, our method is also significantly superior, verifying its effectiveness. More importantly, when evaluated with a larger model \texttt{Llama2-13B}, \texttt{Personai} can still significantly outperform all baseline methods, showing its effectiveness with large models. We also note that both baseline methods significantly decrease under different thresholds, indicating their limited generalization for different discriminators. In comparison, our method is significantly better under different thresholds and is also more stable under different datasets. We show more details in Appendices~\ref{app:exp-ablation} and \ref{appdx:rebuttal}.

\textbf{\texttt{Llama3} Family.} In addition, more results on the \texttt{Llama3} family, i.e., \texttt{Llama3-8B}, \texttt{Llama3-70B}, \texttt{Llama3-80B}, \texttt{Llama3-70B-chat}, and \texttt{Llama3-80B-instruct} show that \texttt{Personai} is also superior to other baseline methods. More importantly, the recall values of baseline methods are still low, indicating their limited effectiveness. Therefore, the effectiveness of \texttt{Personai} can be well demonstrated with different models from the \texttt{Llama3} family. Moreover, these results also verify the effectiveness of \texttt{Personai} with Chat models, i.e., \texttt{Llama3-70B-chat}, and \texttt{Llama3-80B-instruct}. 

\textbf{Other LLMs.} More importantly, in addition to the \texttt{Llama} family, we also evaluate \texttt{Personai} using LLMs trained with different data, i.e., \texttt{Mistral-7B}. 
The results in Table~\ref{tab:main} show that \texttt{Personai} is still significantly outperforming other methods by a large margin. More importantly, the results demonstrate the effectiveness of \texttt{Personai} when tested with different LLMs.

\textbf{\texttt{MMLL}.} In addition to \texttt{RedQueen}, we also demonstrate the effectiveness of \texttt{Personai} on \texttt{MMLL} with four LLMs, i.e., \texttt{Llama2-7B}, \texttt{Llama2-13B}, \texttt{Llama3-8B}, and \texttt{Mistral-7B}. More specifically, we use two guide prompt methods, i.e., ``As a'' and ``As an'', as shown in Table~\ref{tab:mmll}. We observe that all baseline methods significantly outperform their counterparts, with \texttt{Personai} achieving average Recall, Accuracy, and Precision of $0.251, 0.115, 0.165, 0.166$ respectively. In comparison, the best baseline \citet{bai2024} can only achieve $0.021, 0.226, 0.347, 0.309$ respectively. We find that our method is significantly better than other baselines, with accuracy increasing by up to $58.7\%$ when compared to the best baseline. More importantly, accuracy and precision are significantly higher with our method, which may be due to that other methods being not robust for general harmful datasets, as they are targeted to activate harmful personas. More details can be found in the Appendices \ref{app:exp-detailed}.

\begin{table}[!t]
\centering
\caption{Sycophancy Detection Evaluation on \texttt{MMLL}~\cite{gao2023}.}
\label{tab:mmll}
\resizebox{0.49\textwidth}{!}{
\begin{tabular}{l|ccccc}
\Xhline{1.0pt}
   \textbf{Guide}  & F1 & Accuracy @$0.1$ & Accuracy @$0.3$ & Accuracy @$0.5$ & Avg. \\
\hline  
\multicolumn{6}{c}{\texttt{Llama2-7B}} \\
\hline
 \texttt{Cauva} & 0.142 & 0.072 & 0.086 & 0.089 & 0.088 \\
\texttt{Cauva2} &\cellcolor{lightstocust}0.174 & \cellcolor{lightstocust}\textbf{0.094} & \cellcolor{lightstocust}0.101 & \cellcolor{lightstocust}0.110 & \cellcolor{lightstocust}0.105 \\
\hline
\texttt{Cauva+} & 0.187 & 0.100 & 0.116 & 0.091 & 0.110 \\
\texttt{Cauva2+} & 0.212 & 0.121 & \textbf{0.147} & 0.217 & 0.165 \\
\hline 
\multicolumn{6}{c}{\texttt{Llama2-13B}} \\
\hline
\texttt{Cauva} & 0.112& 0.064 & 0.090 & 0.082 & 0.082 \\
\texttt{Cauva2}& 0.136 & 0.087 & 0.092 & 0.095 & 0.094 \\
\hline
\texttt{Cauva+} &0.132 &0.081 & 0.091 & 0.102 & 0.097\\
\texttt{Cauva2+} & 0.176& 0.114 & 0.135 & 0.217 & 0.157 \\
\hline
\multicolumn{6}{c}{\texttt{Llama3-8B}} \\
\hline
\texttt{Cauva} & 0.177 & 0.104 & 0.121 & 0.117 & 0.116 \\
\texttt{Cauva2} &0.177 & 0.111 & 0.131 & 0.129 & 0.126 \\
\hline
\texttt{Cauva+} & 0.241 & 0.147 & 0.158 & 0.189 & 0.170\\
\texttt{Cauva2+} & 0.247 & \textbf{0.155} & 0.212 & 0.244 & 0.209 \\
\hline
\multicolumn{6}{c}{\texttt{Mistral-7B}} \\
\hline
\texttt{Cauva} &0.059 &0.029 & 0.051 & 0.054 & 0.047 \\
\texttt{Cauva2} & 0.069 & 0.037 & 0.059 & 0.069 & 0.056 \\
\hline
\texttt{Cauva+}& 0.076 & 0.049 & 0.057 & 0.076 & 0.063 \\
\texttt{Cauva2+}& \textbf{0.111} & \textbf{0.068} & \textbf{0.091} & \textbf{0.109} & \textbf{0.089} \\
\hline\hline
\end{tabular}
}
\end{table}

\subsection{Ablation Study and Analysis}

\textbf{Guide Samples.} We first conduct an ablation study of guide samples to evaluate their impact on guide prompts for sycophancy detection. The results in Table~\ref{tab:ablation} show that both guide prompts can facilitate \texttt{Personai} in Persona Detection. More importantly, a customized guide sample set is more helpful for improving the performance. 

\begin{table}[!t]
\centering
\caption{Ablation Study. \texttt{Cauva+} and \texttt{Cauva2+} are two guide prompt types.}
\label{tab:ablation}
\resizebox{0.49\textwidth}{!}{
\begin{tabular}{c|c|cccc}
\Xhline{1.0pt}
   \multirow{2}{*}{\textbf{Guide}} &\multirow{2}{*}{\textbf{F1}} & \multicolumn{4}{c}{\textbf{Accuracy}} \\
   \cline{3-6}
& & @$0.1$ &  @$0.3$ & @$0.5$ & Avg. \\
\hline
\texttt{Cauva} & 0.168& 0.108 & 0.14 & 0.149 & 0.134 \\
\texttt{Cauva2} & 0.178 & 0.118 & 0.176 & 0.194 & 0.156 \\
\hline
\texttt{Cauva+}& 0.21 & \cellcolor{lightstocust}0.136 & 0.128 & 0.178 & 0.15\\
\texttt{Cauva2+}& 0.232 & 0.152 & 0.198 & 0.21 & 0.172 \\
\hline
\texttt{Cauva+}* & 0.212 & 0.144 & 0.188 & 0.212 & 0.173 \\
\texttt{Cauva2+}* & \cellcolor{lightstocust}\textbf{0.241} & \cellcolor{lightstocust}\textbf{0.166} & \cellcolor{lightstocust}\textbf{0.218} & \cellcolor{lightstocust}\textbf{0.234} & \cellcolor{lightstocust}\textbf{0.192} \\
\hline\hline
\end{tabular}
}
\end{table}

\textbf{Guide Prompt.} In addition to guide samples, we also demonstrate the results of guide prompts. We expand the initial PAE pool with six different guide prompts, i.e., ``As a [persona]. Is AI chat an appropriate channel for [content]?'' where [persona] is a role prompt, e.g., AI, human, customer. More specifically, the persona of ``AI'' is different from the persona of ``customer''. Therefore, the guide prompts generated may cover different aspects of persona that a target sample contains. In this study, we demonstrate two guide prompts ``As a AI'' and ``As an AI'' as the two guide prompts. More importantly, these two guide prompts significantly facilitate \texttt{Personai} in achieving better persona detection, showing its effectiveness.  

\textbf{Guide Type.} In addition to guide prompts, in Table~\ref{tab:ablation}, we show the results of using different guide samples. We find that customized guide samples are more helpful for improving the performance. More importantly, in general, both guide prompts can facilitate \texttt{Personai} in improving persona detection, showing its effectiveness. Details of the guide prompts are presented in Appendices~\ref{app:guide}.

\textbf{Resource Analysis.} We note that traditional data filtering at the level of data nodes~\cite{bai2024} requires re-ranking for each sample in the training dataset, which may not be scalable when the dataset is large. In comparison, the progressive PAE expansion method in our method (\texttt{Personai}) is not a data filtering method, but more of a data selection method, by selecting samples with diverse prompts to explore more harmful data. More importantly, the number of samples to be used is not significant. Therefore, \texttt{Personai} does not require re-ranking for each sample, making it the most scalable method in terms of dataset size, which is verified in Table~\ref{tab:tcoe}. In addition, we also analyze its Total Cost of Exploration (TCOE)~\cite{gilmer2017few} as a metric of the number of instances of LLMs employed in the data filtering algorithm. We find that the TCOE and inference calls of \texttt{Personai} are limited because 1) it does not require re-ranking for each sample; 2) its guide model and data are considered accessible (stored in the dataset), or if they are not, we may use a much smaller pretrained model to generate harmful data. Therefore, the resource requirements of \texttt{Personai} are significantly low.

\begin{table}[!t]
\centering
\caption{Total cost of exploration (TCOE)~\cite{gilmer2017few} and total inference calls for data filtering shown in the training dataset.}
\label{tab:tcoe}
\resizebox{0.40\textwidth}{!}{
\begin{tabular}{c|ccc}
\Xhline{1.0pt}
   \multicolumn{1}{l|}{\texttt{Method}} & scalability & TCOE & Inst. Calls \\ 
\Xhline{1.0pt}
\texttt{bai2024} & 1 & 1 & $1\times$(x) \\ 
\texttt{Toxicity Detection} & 1 & 1 & $1\times$(x) \\ 
\texttt{Personai} & $\infty$ & $2\times$(x)+$2\times$(M(x))$ & $1\times$(x)+$2\times$(M(x))$ \\ 
\hline\hline
\end{tabular}
}
\end{table}

\textbf{Enchainment Method.} In addition, the effectiveness of enchainment generation is verified in Figure~\ref{fig:exp-progressive}, with the expansion of 100 PAEs. More specifically, in each step, we expand an initial PAE pool with 100 PAEs, and in each step, we sample one guide sample and generate a new PAE and add it into the PAE pool. Using the guide sample set annotated using existing PAEs, we thus do not need human effort in guide sample annotation. Therefore, \texttt{Personai} is highly scalable. 

As shown in Figure~\ref{fig:exp-progressive}, we observe that (1) the PAE pool size increases with each step, shown by its progressive growth due to the enchainment effect of sycophancy. More importantly, we also find that (2) \texttt{Personai} yields accurate guide PAEs, with guide expansion accuracy reaching $99\%$ for \texttt{Llama2-7B} and $98\%$ for \texttt{Llama3-8B}. Therefore, the guide samples are annotated correctly, significantly improving the expansion of PAEs.

\begin{figure}[!t]
\centering
\includegraphics[width=0.33\textwidth]{figs/rebuttal/generation_7b.png}\!\!\!\!\!
\includegraphics[width=0.33\textwidth]{figs/rebuttal/generation_8b.png}
\caption{Iteratively expanded PAE pool size.}
\label{fig:exp-progressive}
\end{figure}

\textbf{Robustness Analysis.} We conduct the same experiments with the guide sample set replaced with guide samples of different types, including random texts from the web, random samples from the training dataset, images, spelling mistakes, and positive samples from the test dataset, as shown in Appendices~\ref{app:exp-robst} in Figure~\ref{fig:exp-robst}. We find that the results on test data are not significantly affected, which also demonstrates the robustness of \texttt{Personai} to diverse guide samples.

\begin{figure}[!t]
\centering
\includegraphics[width=0.33\textwidth]{figs/analysis/ab_study_7b.png}\!\!\!
\includegraphics[width=0.33\textwidth]{figs/analysis/ab_study_8b.png}
\caption{Accuracy of \texttt{Personai} with different guide samples.}
\label{fig:exp-robst}
\end{figure}



\section{Conclusion}

We propose \texttt{Personai}, a novel approach for persona detection and mitigation. In response to the challenge of detecting harmful persona influences in LLMs, \texttt{Personai} formulates the persona detection problem as a prompt retrieval task, which can also mitigate persona influences without data filtering. More significantly, \texttt{Personai} is motivated by two key findings, by which persona detection can be better addressed at paths (rather than data nodes). The experiment results show that \texttt{Personai} significantly outperforms baseline methods in both effectiveness and scalability, with performance verified across different LLMs and datasets. In general, our findings may also shed new light on future works on persona detection.

\textbf{Reproducibility. All experiments can be found in the Appendices, i.e., Appendices~\ref{app:exp-detailed},~\ref{app:exp-ablation}, and~\ref{appdx:rebuttal}. 


\textbf{Ethics. Our work may contribute to the field by detecting and thus may contribute to addressing social issues caused by sycophancy in LLMs, such as hallucination, biases, and poor reasoning capabilities~\cite{he2023, hu2024, hu2025}. 





\section{Disclosure}
